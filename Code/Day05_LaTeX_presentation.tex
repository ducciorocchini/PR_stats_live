\documentclass{beamer}
\usepackage{graphicx} % Required for inserting images
\usepackage{hyperref}

\usetheme{Frankfurt}
\usecolortheme{dove}

\title{PR-Stats presentation}
\author{Duccio Rocchini}
\date{June 25th 2025}

\institute{Talk held at:\\
    \bigskip
    \centering
    \includegraphics[width=0.5\linewidth]{esa.png}
           }

\begin{document}

\maketitle

\AtBeginSection[] 
{
\begin{frame}{Outline}
\tableofcontents[currentsection]
\end{frame}
}

\section{Introduction}

\begin{frame}{First slide}
    This is the first slide I am making in LaTeX.
\end{frame}

\begin{frame}{My second slide}
    This is the first slide I am making in LaTeX. Let me take you down
'Cause I'm going to \textbf{strawberry fields}
Nothing is real
And nothing to get hung about
Strawberry \textbf{fields} forever
Living is easy with eyes closed
Misunderstanding all you see
It's getting hard to be someone, but it all works out
It doesn't matter much to me,
\end{frame}

\begin{frame}{Itemization}
    In the following text I will put some items like:
    \begin{itemize}
        \item Nothing is real
        \item Living is easy with eyes closed
        \item Misunderstanding all you see
    \end{itemize}
\end{frame}

\begin{frame}{Pausing}
    In the following text I will put some items like:
    \begin{itemize}
        \item Nothing is real
        \item \pause Living is easy with eyes closed
        \item \pause Misunderstanding all you see
    \end{itemize}
\end{frame}

\begin{frame}{Pausing}
    In the following text I will put some items like:
    \begin{enumerate}
        \item Nothing is real
        \item \pause Living is easy with eyes closed
        \item \pause Misunderstanding all you see
    \end{enumerate}
\end{frame}

\section{Algorithm}

\begin{frame}{The basic formula}
In this study we made use of the Shannon index:
    \begin{equation}
        H = - \sum_{i=1}^N p_i \times \ln p_i
    \end{equation}
where $N$=number of classes, $p$=proportion, $i$=$i$th class.

In order to also consider evenness we also applied the Pielou index:
    \begin{equation}
        J = \frac{H}{H_{max}}
    \end{equation}
\tiny{Note: For the code, check my GitHub account here: \url{https://github.com/ducciorocchini/PR_stats/blob/main/Code/Day5_LaTeX_presentation.tex}}
\end{frame}

\begin{frame}{Code}
For this study we made use of the cblindplot package with the following function:\\
\bigskip
\texttt{cblind.plot(image, cvd='protanopia')} \\
\bigskip
This would allow colorblind people to look at colors in maps.
\end{frame}

\section{Final achievements}
\begin{frame}{Putting images}
        \centering
        \includegraphics[width=.9\linewidth]{fancy.png}
\end{frame}

\begin{frame}{Putting images}
        \centering
        \includegraphics[width=.4\linewidth]{fancy.png}
        \includegraphics[width=.4\linewidth]{fancy.png}  
\end{frame}

\begin{frame}{Putting images}
        \centering
        \includegraphics[width=.4\linewidth]{fancy.png}
        \includegraphics[width=.4\linewidth]{fancy.png}\\  
        \includegraphics[width=.4\linewidth]{fancy.png}
        \includegraphics[width=.4\linewidth]{fancy.png}  
\end{frame}

\section{Conclusion}
\begin{frame}{Coda}
Concluding remarks:
    \begin{itemize}
        \item Nothing is real. Nothing is real. Nothing is real. Nothing is real. Nothing is real.
        \item \pause Living is easy with eyes closed. Nothing is real. Nothing is real. Nothing is real. Nothing is real.
        \item \pause Misunderstanding all you see. Nothing is real. Nothing is real. Nothing is real. Nothing is real. Nothing is real.
    \end{itemize}
\end{frame}

\end{document}
